
%
% abstract
%


\chapter*{Abstract}

This thesis investigates ways of visualization for presenting clustered information on maps. Performance and readability of digital mapping applications decreases when displaying large amounts of data. Clustering can be used to group overlapping items and reduce visual clutter of the map presentation. 

Visualization techniques for putting clusters on a map are researched and evaluated in an exploratory analysis. Existing concepts from visual cluster analysis and geovisualization, as well as established classifications form the foundations of the presented study. \textit{Map types} as well as \textit{cluster visualization techniques} are presented. The \textit{evaluation} classifies the stated techniques for cluster visualization on maps.

The presented map types for visualizing clusters range from standard, \textit{geographic maps} with markers, \textit{Heat maps} to \textit{Dot Grid maps} and a \textit{Voronoi} map. Cluster visualization techniques include \textit{icon-based/glyphs}, \textit{pixel-based}, as well as \textit{geometric/diagrams}. The evaluation of visualization techniques for clusters on a map proposes a custom set of criteria. \textit{Showing the number of items within clusters} is considered as a main feature of simple cluster visualizations on maps. Further, the study identifies \textit{showing cluster areas} and \textit{extra cluster information} as two additional aspects of primary interest when comparing approaches to cluster visualization on maps.   

The results are discussed and put into practice by evaluating the visual aspects of \textit{Geocluster}, a server-side clustering implementation for maps.